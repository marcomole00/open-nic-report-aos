\documentclass[10pt,a4]{article}
\usepackage[english]{babel}
\usepackage[utf8x]{inputenc}
\usepackage{amsmath}
\usepackage{graphicx}
\usepackage[colorinlistoftodos]{todonotes}
\PassOptionsToPackage{hyphens}{url}\usepackage[hidelinks]{hyperref}
\usepackage{geometry}
\geometry{top=3cm,left=2cm,right=2cm,bottom=3cm}
\usepackage[scaled]{helvet}
\usepackage[T1]{fontenc}
\renewcommand\familydefault{\sfdefault}


\author{Alexandru Gabriel Bradatan, Marco Molè}
\date{\today}
\title{Adding XDP support to Open Nic Driver}



\begin{document}
\maketitle
\tableofcontents



\section{Project data}

\begin{itemize}
\item Project supervisor(s): Gianni Antichi \& Sebastiano Miano
\item Describe in this table the group that is delivering this project:

\begin{center}
\begin{tabular}{lll}
Last and first name & Person code & Email address\\
\hline
  Alexandru Gabriel Bradatan & 10658858 & alexandrugabriel.bradatan@mail.polimi.it \\
  Marco Molè & 10676087  &     marco.mole@mail.polimi.it
\end{tabular}
\end{center}

\item
Describe here how development tasks have been subdivided among members
of the group, e.g.:

\begin{itemize}
\item Most of the code was written in pair programming sessions.
\begin{itemize}
	\item We found the pair programming approach really useful to delve into the XDP API, which sometimes is poorly documented.
\end{itemize}
\item Bradatan worked on a eBPF testing suite that was not deemed necessary at the end.
\begin{itemize}
  \item Link to the tests: \url{https://github.com/alexbradd/ebpf-xdp-test-suite}
\end{itemize}
\end{itemize}

\item Links to the project source code: \url{https://github.com/marcomole00/open-nic-driver/tree/xdp-support}
Note: our work is in the \texttt{xdp-support} branch

\end{itemize}


\section{Project description}

% What are the goals
% Why it is important to the AOS course?
\subsection{Project goals}
The goal of this project is to add support for \textit{eXpress Data Path} (XDP)
for the driver of the \textbf{AMD OpenNIC project}.

XDP is a high-performance data path for network packets present in the Linux
kernel since version 4.8. It enables users to install packet processing programs
in the kernel that are executed before the invocation of the standard network stack.
These programs are written and executed in eBPF, a subset of C that can be
verified to terminate and to not corrupt kernel memory making it safe to use for
extending the kernel's functionality.



\subsection{Importance to the AOS course}
This project is relevant to the Advanced Operating System course because it
requires extending the functionality of a Linux driver. Doing so requires
understanding how Linux kernel modules are written and compiled, both of which
are topics of the course.

Moreover, implementing XDP requires knowledge of the Linux networking
stack, which even though is not a topic of the AOS course but of
the supervisors' Netowrk Computing course, it is still an important part of the
Linux kernel.



\subsection{Design and implementation}
%Describe here the structure of the solution you devised. Note, don't put major
%parts of the source code here; if you can, put hyperlinks to existing repos.

%For those who choose to work on an open source project, please put here an
%history (mail messages/github issues etc..) of the interaction with the
%development team that helped you identify such design and the code reviews that
%helped you improve it.

% Those that have chosen to present a paper can exclude this section.

We based our implementation based on some previous work done on the Intel
driver~\footnote{\url{https://patchwork.ozlabs.org/project/intel-wired-lan/patch/20200902203222.185141-1-anthony.l.nguyen@intel.com/\#2535008}}
and some documentation material provided by RedHat~\footnote{\url{https://people.redhat.com/lbiancon/conference/NetDevConf2020-0x14/add-xdp-on-driver.html}}.

Our work is mainly focused in \texttt{onic\_netdev.c}, where we added all the
\texttt{onic\_xdp\_\*} functions, the most relevant of which are:

\begin{itemize}
  \item \texttt{onic\_xdp()}: entry-point of for the XDP subsystem, does some
    initialization
  \item \texttt{onic\_run\_xdp()}: main function that executes the XDP program
    on each received packet and handles the return value of the program; invoked
    for each packet by the main driver polling loop in \texttt{onic\_poll()}
  \item \texttt{onic\_xdp\_xmit()}: used to submit XDP packets for transmit
\end{itemize}

We have also added support for querying XDP statistics through \texttt{ethtool}.
Those changes have been done mainly to the \texttt{onic\_ethtool.c} file.


\section{Project outcomes}

\subsection{Concrete outcomes}
%Describe the artifacts you've produced, if possible by linking to repo commits.
%For those who choose to work on an open source project, please put here the
%\textbf{URL to your final pull request}.
%
%Those that have chosen to present a paper can include a link to the slides.

We have produced a series of commits on a fork of the OpenNIC repository
implementing XDP and a repo containing some tests for the various return
values of XDP programs.

Links:

\begin{itemize}
  \item Commits implementing XDP: \url{https://github.com/marcomole00/open-nic-driver/commits/xdp-support/}
  \item Diff w.r.t. upstream: \url{https://github.com/Xilinx/open-nic-driver/compare/main...marcomole00:open-nic-driver:xdp-support}
  \item Testing repo: \url{https://github.com/alexbradd/ebpf-xdp-test-suite}
\end{itemize}

\subsection{Learning outcomes}
% What was the most important thing all the members have learned while
% developing this part of the project, what questions remained unanswered,
% how you will use what you've learned in your everyday life?
% Please also indicate which tools you learned to use.
\begin{itemize}

\item We gained insights on how the linux kernel code is structured and how to navigate it
effectively when searching for a specific function. We made extensive use of the \textit{Elixir Cross Referencer}~\footnote{\url{https://elixir.bootlin.com}}.

\item We learned to search in commit messages or mailing list archives when in doubt about certain pieces of code.
\item We learned how to correctly setup the environment for kernel module programming.

\end{itemize}


% Examples:

% \begin{itemize}
% \item Foo learned to write multithreaded applications, he's probably going to
%   create his own startup with what she has learned. She also learned how to
%   debug with gdb.
% \item Bar learned how to interact with the open source community, politely
%   answering to code reviews and issuing pull requests through Git.
% \end{itemize}

\subsection{Existing knowledge}
%What courses you have followed (not only AOS) did help you in doing this project
%and why? Do you have any suggestions on improving the AOS course with topics
%that would have made it easier for you?

Bradatan followed profs. Antichi and Miano's Network Computing course last year,
Molè is following it this semester. The course provides insights on how the Linux
network stack is implemented.

This project focused on some topics taught in the Network Computing course. To
avoid increasing too much the scope of the AOS course and to avoid overlap with
the Network Computing course we do not feel like recommending adding networking
topics to the AOS course. Moreover, profs. Antichi and Miano were very available
to answering any questions and providing course materials one the topics needed
for the project.

\subsection{Problems encountered}
%What were the most important problems and issues you encountered? Did you ever
%encountered them before?
%
%\begin{itemize}
%\item Foo encountered a problem with some critical sections. He ended up
%  rewriting existing lock implementation.
%\end{itemize}

The main problem we encountered was the lack of (or when present
poor quality of) documentation on some parts of the linux networking stack and
on OpenNIC driver code. This required some hunting around sources and
cross-referencing with other driver's implementations to get resolved.

Another thing we had to consider is that the XDP API and the network stack data structures changed
over time and that led to some inconsistency in kernel versions previous to 5.3.0 that hindered us from
implementing the \textit{onic\_xdp\_xmit\_frame} function, and thus the support
for \texttt{XDP\_TX} and \texttt{XDP\_REDIRECT} in those older versions.

\section{Honor Pledge}
(\textbf{This part cannot be modified and it is mandatory to sign it})

I/We pledge that this work was fully and wholly completed within the criteria
established for academic integrity by Politecnico di Milano (Code of Ethics and
Conduct) and represents my/our original production, unless otherwise cited.

I/We also understand that this project, if successfully graded,  will fulfill part B requirement of the
Advanced Operating System course and that it will be considered valid up until
the AOS exam of Sept. 2024.

\begin{flushright}
Group Students' signatures
\end{flushright}


\end{document}
