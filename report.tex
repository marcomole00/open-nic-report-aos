
\documentclass[10pt,a4]{article}
\usepackage[english]{babel}
\usepackage[utf8x]{inputenc}
\usepackage{amsmath}
\usepackage{graphicx}
\usepackage[colorinlistoftodos]{todonotes}
\usepackage{hyperref}
\usepackage{geometry}
\geometry{top=3cm,left=2cm,right=2cm,bottom=3cm}
\usepackage[scaled]{helvet}
\usepackage[T1]{fontenc}
\renewcommand\familydefault{\sfdefault}


\author{Alexandru Gabriel Bradatan, Marco Molè}
\date{\today}
\title{Adding XDP support to Open Nic Driver}



\begin{document}
\maketitle
\tableofcontents



\section{Project data}

\begin{itemize}
\item  Project supervisor(s): Gianni Antichi \& Sebastiano Miano 

\item 
Describe in this table the group that is delivering this project:

\begin{center}
\begin{tabular}{lll}
Last and first name & Person code & Email address\\
\hline
  Alexandru Gabriel Bradatan & 0101010 & alexandrugabriel.bradatan@mail.polimi.it \\
  Marco Molè & 10676087  &     marco.mole@mail.polimi.it        
\end{tabular}
\end{center}

\item
Describe here how development tasks have been subdivided among members
of the group, e.g.:

\begin{itemize}
\item Most of the code was written in pair programming sessions.
\begin{itemize}
	\item We found the pair programming approach really useful to delve into the XDP API, which is documented a bit poorly. 
\end{itemize}
\item Bradatan worked on a eBPF testing suite that was not deemed necessary at the end.
\end{itemize}

\item Links to the project source code:  \url{https://github.com/marcomole00/open-nic-driver/tree/xdp-support}
Note: our work is in the \textbf{xdp-support} branch

\end{itemize}


\section{Project description}

% What are the goals
% Why it is important to the AOS course? 
\subsection{Project goals}
The goal of this project is to implement the \textit{eXpress Data Path} (XDP) for the driver of the \textbf{AMD OpenNIC project}.

XDP is a high-performance  data path for network packets present in the Linux kernel since version 4.8. It enables users to install packet processing programs in the kernel before the invocation of the standard network stack. 

These programs are written and executed in eBPF which is a language/runtime for extending operating systems.



\subsection{Importance to the AOS course}
This project is relevant to the Advanced Operating System course because it involves extending the functionality of a Linux driver, which is an argument of the course.



\subsection{Design and implementation}
%Describe here the structure of the solution you devised. Note, don't put major
%parts of the source code here; if you can, put hyperlinks to existing repos.

%For those who choose to work on an open source project, please put here an
%history (mail messages/github issues etc..) of the interaction with the
%development team that helped you identify such design and the code reviews that
%helped you improve it.

% Those that have chosen to present a paper can exclude this section. 



\section{Project outcomes}

\subsection{Concrete outcomes}
Describe the artifacts you've produced, if possible by linking to repo commits.
For those who choose to work on an open source project, please put here the 
\textbf{URL to your final pull request}.

Those that have chosen to present a paper can include a link to the slides.

\subsection{Learning outcomes}

What was the most important thing all the members have learned while
developing this part of the project, what questions remained unanswered,
how you will use what you've learned in your everyday life?
Please also indicate which tools you learned to use.

Examples:

\begin{itemize}
\item Foo learned to write multithreaded applications, he's probably going to
  create his own startup with what she has learned. She also learned how to
  debug with gdb.
\item Bar learned how to interact with the open source community, politely
  answering to code reviews and issuing pull requests through Git.
\end{itemize}

\subsection{Existing knowledge}
What courses you have followed (not only AOS) did help you in doing this project
and why? Do you have any suggestions on improving the AOS course with topics
that would have made it easier for you?

\subsection{Problems encountered}
What were the most important problems and issues you encountered? Did you ever
encountered them before? 

\begin{itemize}
\item Foo encountered a problem with some critical sections. He ended up
  rewriting existing lock implementation.
\end{itemize}

\section{Honor Pledge}
(\textbf{This part cannot be modified and it is mandatory to sign it})

I/We pledge that this work was fully and wholly completed within the criteria
established for academic integrity by Politecnico di Milano (Code of Ethics and
Conduct) and represents my/our original production, unless otherwise cited.

I/We also understand that this project, if successfully graded,  will fulfill part B requirement of the
Advanced Operating System course and that it will be considered valid up until
the AOS exam of Sept. 2022. 

\begin{flushright}
Group Students' signatures
\end{flushright}


\end{document}
